\documentclass[pageno]{styles/jpaper}

\usepackage[normalem]{ulem}
\usepackage{listings}
\usepackage{color}
\usepackage{xspace}
\usepackage[title]{appendix}


%replace XXX with the submission number you are given from the ASPLOS submission site.
\newcommand{\asplossubmissionnumber}{XXX}
\newcommand{\revisit}[1]{{\color{red} #1}}
\newcommand{\cmt}[1]{}


\newcommand{\Strata}{{\tt Strata}\xspace}
\newcommand{\Stoke}{{\tt Stoke}\xspace}
\newcommand{\K}{{\tt K}\xspace}
\newcommand{\TS}[1]{{\tt #1}}
\newcommand{\instr}[1]{\texttt{#1}}
\newcommand{\opcode}[1]{\ensuremath{#1}}
\newcommand{\cond}[1]{\ensuremath{#1}}
\newcommand{\extract}{\emph{extract}\xspace}
\newcommand{\extractMInt}{\emph{extractMInt}\xspace}
\newcommand{\false}{\textbf{False}}
\newcommand{\true}{\textbf{True}}
\newcommand{\bool}{\texttt{Bool}\xspace}
\newcommand{\bv}{\texttt{Bitvector}\xspace}
\newcommand{\fig}[1]{\includegraphics[scale=.5]{#1}}




\definecolor{codegreen}{rgb}{0,0.6,0}
\definecolor{codegray}{rgb}{0.5,0.5,0.5}
\definecolor{codepurple}{rgb}{0.58,0,0.82}
\definecolor{backcolour}{rgb}{0.95,0.95,0.92}
 
\lstdefinestyle{mystyle}{
    backgroundcolor=\color{backcolour},   
    commentstyle=\color{codegreen},
    keywordstyle=\color{magenta},
    numberstyle=\tiny\color{codegray},
    stringstyle=\color{codepurple},
    basicstyle=\footnotesize,
    breakatwhitespace=false,         
    breaklines=true,                 
    captionpos=b,                    
    keepspaces=true,                 
    numbers=left,                    
    numbersep=5pt,                  
    showspaces=false,                
    showstringspaces=false,
    showtabs=false,                  
    tabsize=2
}
\lstset{style=mystyle}

\begin{document}

\title{
Near Complete Formal Semantics of X86-64}

\date{}
\maketitle

\thispagestyle{empty}

\begin{abstract}

\revisit{ToDo}

\end{abstract}

\section{Introduction}\label{sec:Intro}

\section{Challenges}

\subsection{Using Strata Results}
 Following are the challenges in using \Strata{}~\cite{Heule2016a} (or \Stoke{}) 
 formula as is.    
 
\begin{itemize}

  \item

    \Stoke uses C+-functions which define the semantics of instructions. For
    example, following is the function to define the semantics of add
    instruction. The functions are generic in the sense that they can be used to
    obtain obtain the concrete semantics of any instruction like \texttt{add\
      \%rax,\ \%rbx}
  
\begin{lstlisting}[language=C]

void add(SymBitVector dest, SymBitVector a, SymBitVector b) {
  set(d, a+b);
}
  \end{lstlisting}
  
  The untested assumption here is the generic formula will behave identically
  for all the variants. We have tested all the formula for each instruction
  variant. 
  
  \item 

  \Strata gives the concrete semantics for a concrete instructions. For other
  variants it generalize from the concrete semantics. Assumption is the
  generalization is correct. Test all the generalization.

  \item 

While porting to \K rule, we generalize the from a concrete semantics that
strata provides. Is this generalization faithful? For instruction like
\instr{xchg, xadd, cmpxchg}, the formula is different for different
operands. So the general \K rule we obtain from \instr{xchgl a, b} may not
represent the semantics for \instr{xchgl a, a}. Fortunately there exists
different instruction variants if the their semantics might be different and
accordingly we might have different \K rules. For example,
            \opcode{xchgl\_r32\_eax} and \opcode{xchgl\_r32\_r32}. But even for
            \opcode{xchgl\_r32\_32} semantics could be different for cases
            \cond{r1\ !=r2} and \cond{r1\ ==\ r2}. Idea: Once lifted as \K
            rule, test the instruction for all variants.  
            
\end{itemize}

Lets consider \instr{xaddb SRC, DEST}, as per manual the semantics is
as follows:

{\small \tt 
  \begin{tabular}[b]{l}
   S1. Temp = Src + Dest \\ 
   S2. Src  = Dest \\
   S3. Dest = Temp \\
  \end{tabular}
}

\cmt{
\begin{eqnarray*}
  \TS{Temp} &=& \TS{Src} + \TS{Dest} \\
  \TS{Src} &=& \TS{Dest} \\
  \TS{Dest} &=& \TS{Temp} \\
\end{eqnarray*}
}

The point to note here is that the register updates follow an order.  \Strata
uses \instr{xaddb \%rax, \%rbx}, to obtain the semantics and it happened that
the ordering is maintained and hence strata can generalize the semantics of
\instr{xaddb R1, R1}. But even if the ordering is not maintained the
semantics is going to be the same for the case \cond{R1 != R2}, but the generalization
for the \texttt{R1\ ==\ R1} case will mess up.  We cannot trust the above
generalization by strata. We need to test the \K rule for all possible operands.


\section{Modeling X86-64 Instruction Semantics} \label{sec:modelI}

For the purpose we used project \Strata, which automatically synthesized formal semantics  of the input/output behavior for $1796$ Haswell ISA X86-64 instructions. The key to their results is stratified synthesis, where they use a set of instructions whose semantics are known to synthesize the semantics of additional instructions whose semantics are unknown. Using the technique, they 
come up with the semantics of $692$ register variants and $\sim120$ immediate variants. The rest $\sim984$ are obtained by generalizing the register variants to memory and immediate variants.     

We borrowed the semantics of $692$ register variants. We answered the following questions to ourself before borrowing the rest.   

\begin{itemize}
    \item How reliable is the generalization of register variants to memory or immediate variants?   
    
    \item  For immediate  variants that do not have a corresponding register-only instruction, \Strata  learns a separate formula for every possible value of the constant provided the constant value is of width 8 bits. Also in some cases, they learned a formula only for some of the possible constants.
    
    In order to have a more intuitive generic semantics of those instructions, the relevant question is: How to get a generic formula for an immediate instruction for which we have separate formulas for all or a subset of constant values?   
\end{itemize} 
\ \\
\Strata covers a splendid number of $61.5\%$ of the instructions in scope. For our purpose, we aim to cover the rest. Following are the observations and conclusions that help us formulate a strategy to achieve the goal.

\paragraph{Observation \& Conclusion}
     In order to get the semantics of a target instruction {\tt I}, \Strata uses \Stoke along with set {\tt TS} of $6580$ test cases to synthesize an instruction sequence which agrees with {\tt I} on {\tt TS} (which means the output behavior of the instruction sequence matches with the real hardware execution on input {\tt TS}). After having that \emph{initial search}, they keep on searching  additional sequences (which they call {\tt secondary searches} each agreeing with {\tt I} on {\tt TS}) in a hope of getting  one which would prove non-equivalent to existing ones and thereby gaining more confidence on the search and probably a better test-suite (as {\tt TS} might get augmented with a counter example from equivalence checker in the event of non-equivalence). One unfavorable possibility for \Strata is when all subsequent secondary search results proves  equivalent to the one obtained from initial search, in which case it  means that  secondary searches fail to add any ``confidence'' to the initial search result and the final outcome of stratification is having the same correctness guarantee as that provided by the initial search result, which is ``correctness over {\tt TS}''. But in those unfavorable case, the secondary searches might have provided ``better'' choices to pick the final formula from. A better choice of formula do not contain uninterpreted functions or  non-linear arithmetics  and are simple.  
    
   In the paper\cite{Heule2016a}, it is mentioned that there are only $50$ cases, where they found a (valid) counterexample. That means, there are $762 = (692 + 120 - 50)$ cases, where the initial search is sufficient enough to be accepted, as all the later secondary searches results are equivalent to the one obtained from the {\tt initial search}. In other words, in  most of the cases, the correctness guarantee of stratification is same as that of the initial search result. 
   
   For the unstratified instructions, we would need a \emph{semantics generator} to provide us with an initial candidate of the instruction semantics. Once we have that semantics, we could test it against hardware on the same augmented test-suite (containing $6630 = 6580 + 50$ ) that \Stoke uses and if the candidate  matches then we can claim to have the same correctness guarantee as above. 
   
   Now the missing piece, the  \emph{semantic generator}, can be  projects like \Stoke, which have manually written instruction semantics (in terms of logical formulas), or can be manually written.  We understand that this is not as efficient as \Stoke, which is fully automatic in getting these formulas, but our contribution is 1. To deliver in cases where \Stoke cannot   2. To cover  as many instruction semantics as possible. Moreover, writing the semantics manually might alleviate the need of secondary search as a means to provide ``better'' formula as we can control the complexity and choice of operations to include in the formula. Also carefully written manual formula tend to need less number of conflicting searches than the onces generated by random search engines like Stoke.
   
  

\subsection{Porting Formulas for stratified instructions to \K Rules}

For the purpose of getting  \K rules, we could have directly converted the \Strata formulas
for an instruction to \K rule assuming that the \Strata's symbolic execution over the 
stratified instruction sequence is correct.

Given that fact the \K's symbolic execution engine is more trusted as 
that has been used extensively in language-agnostic manner to perform symbolic
execution, we decided to use \K's symbolic executor. Also in order to check 
if \Strata's symbolic execution engine is correct, we did an equivalence check on 
the outputs of both the symbolic executions.   
 

\begin{enumerate}
\item Implementing the base instructions semantics in \K and testing them.
\item Symbolic execution of the stratified instruction sequences.
\item Dealing with scratch pad registers.
\item Equivalence check between \Strata formula ad the output of 2.
   
   All the checks are \emph{unsat}, expect one where the check fail to due a bug in the simplification
   rules in \Strata, which states the following lemma related to two single precision floating point numbers  {\tt A}  and {\tt B}, which is not correct for {\tt NaNs}. However this bug is fixed in 
   the latest version of \Stoke. 
   
   
   { \tt  
        \begin{tabular}[b]{l}
   \qquad sub\_single(A, B) $\equiv$ 0 if A == B     
      \end{tabular}
  }
   
\item {Simplification of formulas:}
  Simplification generates simple \K rule (sometimes simpler than the corresponding \Strata formula).
 Also it is much easier to write the simplification rules in
  \K.\revisit{show the example for concat(A[1:2], concate(B[2:3], X)) $\equiv$
    concate(A[1:3], X)}


\item One drawback of the \Strata formulas is they could be non-intuitive and complex at times when the simplification rules are not adequate enough to simplify their complexity to more intuitive formulas. Appendix \ref{sec:AP:A} provides such an example.
Towards the goal of having intuitive formulas, we borrowed the hand written formula (provided they are simpler) from \Stoke or manually write those  and check equivalence with the stratified formula. If they match on all register state and/or memory, we employ that in our \K semantics.

\cmt{
An example of one such simplification opportunity is: 
     { \tt  
      \begin{tabular}[b]{l}
          ($0_{32}$ $\cdot$ \%rax[32:0]) $\oplus$ \%rax $\equiv$ \%rax[63:32] $\cdot$  $0_{32}$ 
      \end{tabular}
    }
}

       


\end{enumerate}

\subsection{Supporting un-stratified instructions \& Porting their formulas to \K Rules}

\subsubsection{Supporting un-stratified instructions}
\paragraph{Instruction support status}

\begin{figure*}[t]
\centering
\fig{figures/instruction_classification}
\caption{Instruction classification\label{fig:instr_class}}
\end{figure*}

\subsection{Porting to \K Rules}

\Strata could output the internal AST, used to model a register state formula, in different
formats. Supported backend are SmtLib and Prefix notation. We have added another backend 
to generate \K rule. We need some way to validate the backend. 

\paragraph{Validate the Backend}

The \K rules generated using the backend are matched (syntactically)  against
the ones we already obtained via symbolic execution on stratified instructions.
Other than validaing the backend, this has an added benefit that in order to get
the exact match, we need to port all the simplification rules from \K to strata
code, which in turn will later help in generating simplified \K rules for
non-stratified instructions. 

Main challenges in getting an exact match are:
\begin{itemize}

\item  \Strata rules uses \extract to extract portion of a bit-vector. The high
and low indices of \extract are obtained considering LSB at index 0, whereas \K
uses \extractMInt for the same purpose, but uses MSB at index zero.

\item  \Strata uses flags as \bool, whereas they are treated as \bv in our
semantics. We modifed strata so as to treat flag registers as 1 bit bitvectors.

\end{itemize}






\begin{appendices}
\section{An Example of \Strata Formula}\label{sec:AP:A}

Following is the \Strata formula for an instruction \instr{vpxor \%ymm3, \%ymm2, \%ymm1},


\begin{lstlisting}[language=Java]

(let ((a!1 (bvxor ((_ extract 255 192) ymm3)
                  ((_ extract 255 192) ymm2)
                  (bvor ((_ extract 255 192) ymm3)
                        (bvxor ((_ extract 255 192) ymm3)
                               ((_ extract 255 192) ymm2)))))
      (a!3 (bvxor ((_ extract 191 128) ymm3)
                  ((_ extract 191 128) ymm2)
                  (bvor ((_ extract 191 128) ymm3)
                        (bvxor ((_ extract 191 128) ymm3)
                               ((_ extract 191 128) ymm2)))))
      (a!5 (bvxor ((_ extract 127 64) ymm3)
                  ((_ extract 127 64) ymm2)
                  (bvor ((_ extract 127 64) ymm3)
                        (bvxor ((_ extract 127 64) ymm3)
                               ((_ extract 127 64) ymm2)))))
      (a!7 (bvxor ((_ extract 63 0) ymm3)
                  ((_ extract 63 0) ymm2)
                  (bvor ((_ extract 63 0) ymm3)
                        (bvxor ((_ extract 63 0) ymm3) ((_ extract 63 0) ymm2))))))
(let ((a!2 (bvxor ((_ extract 255 192) ymm3)
                  ((_ extract 255 192) ymm2)
                  (bvor ((_ extract 255 192) ymm3)
                        (bvxor ((_ extract 255 192) ymm3)
                               ((_ extract 255 192) ymm2)))
                  (bvor a!1
                        ((_ extract 255 192) ymm2)
                        ((_ extract 255 192) ymm3))))
      (a!4 (bvxor ((_ extract 191 128) ymm3)
                  ((_ extract 191 128) ymm2)
                  (bvor ((_ extract 191 128) ymm3)
                        (bvxor ((_ extract 191 128) ymm3)
                               ((_ extract 191 128) ymm2)))
                  (bvor a!3
                        ((_ extract 191 128) ymm2)
                        ((_ extract 191 128) ymm3))))
      (a!6 (bvxor ((_ extract 127 64) ymm3)
                  ((_ extract 127 64) ymm2)
                  (bvor ((_ extract 127 64) ymm3)
                        (bvxor ((_ extract 127 64) ymm3)
                               ((_ extract 127 64) ymm2)))
                  (bvor ((_ extract 127 64) ymm2) ((_ extract 127 64) ymm3) a!5)))
      (a!8 (bvxor ((_ extract 63 0) ymm3)
                  ((_ extract 63 0) ymm2)
                  (bvor ((_ extract 63 0) ymm3)
                        (bvxor ((_ extract 63 0) ymm3) ((_ extract 63 0) ymm2)))
                  (bvor ((_ extract 63 0) ymm2) ((_ extract 63 0) ymm3) a!7))))
  (concat a!2 a!4 a!6 a!8)))
\end{lstlisting}

where as following is the formula obtained from \Stoke (hand-written) and 
 Z3 took $88.70$ secs to prove that they are equivalent.

\begin{lstlisting}[language=Java]
%ymm1  : (bvxor %ymm2 %ymm3)
\end{lstlisting}
\end{appendices}


\bibliographystyle{plain}
\bibliography{bibs/references,bibs/modeling-X86-semantics}


\end{document}

\grid
\grid
