\documentclass[pageno]{jpaper}

%replace XXX with the submission number you are given from the ASPLOS submission site.
\newcommand{\asplossubmissionnumber}{XXX}

\usepackage[normalem]{ulem}

\begin{document}

\title{
Instructions for Submission to ASPLOS 2018}

\date{}
\maketitle

\thispagestyle{empty}

\begin{abstract}

This document is intended to serve as a sample for submissions to the
23rd International Conference on Architectural Support for Programming
Languages and Operating Systems (ASPLOS), 2018.  It provides
guidelines that authors should follow when submitting papers to the
conference.

\end{abstract}

\section{Introduction}

This document provides instructions for submitting papers to the 23rd
International Conference on Architectural Support for Programming
Languages and Operating Systems (ASPLOS), 2018.  In an effort to
respect the efforts of reviewers and in the interest of fairness to
all prospective authors, we request that all submissions to ASPLOS
2018 follow the formatting and submission rules detailed below.
Submissions that violate these instructions may not be reviewd, at the
discretion of the program chair, in order to maintain a review process
that is fair to all potential authors.

An example submission (formatted using the ASPLOS'18 submission
format) that contains the submission and formatting guidelines can be
downloaded from here:
\href{https://www.asplos2018.org/wp-content/uploads/2017/07/asplos18-template.pdf}{Sample PDF}. The content of
this document mirrors that of the submission instructions that appear
on \href{https://www.asplos2018.org/submissions/}{this
website}, where the paper submission site will be linked online
shortly.

All questions regarding paper formatting and submission should be directed
to the program chair.

\paragraph{Highlights ({\bf note the following})}:
\begin{itemize}
\item Paper must be submitted in printable PDF format.
\item Text must be in a minimum 10pt ({\bf not} 9pt) font.
\item Papers must be submitted in printable PDF format and should contain a
maximum of 11 pages of single-spaced two-column text, including any
appendixes, but not including references.
\item No page limit for references.
\item Each reference must specify {\em all} authors (no {\em et al.}).
\item Authors may optionally suggest reviewers.
\item Authors of {\em all} accepted papers will be required to give a
lightning presentation (about 90s) and a poster in addition to the regular
conference talk.
\item Proceedings will appear in the ACM digital library up to two weeks
before the conference.
\end{itemize}

\paragraph{Paper evaluation objectives}:

The committee will make every effort to judge each submitted paper on
its own merits. There will be no target acceptance rate.  We expect to
accept a wide range of papers with appropriate expectations for
evaluation --- while papers that build on significant past work with
strong evaluations are valuable, papers that open new areas with less
rigorous evaluation are equally welcome and especially encouraged.
Given the wide range of topics covered by ASPLOS, every effort will be
made to find expert reviewers, including providing the ability for
authors' to suggest additional reviewers.

\section{Paper Preparation Instructions}

\subsection{Paper Formatting}

Papers must be submitted in printable PDF format and should contain a
{\bf maximum of 11 pages} of single-spaced two-column text, including any
appendixes, but {\bf not
  including references}.  You may include any number of pages for
references, but see below for more instructions.  If you are using
\LaTeX~\cite{lamport94} to typeset your paper, then we suggest that
you use the template here:
\href{https://www.asplos2018.org/wp-content/uploads/2017/07/asplos18-latex-template.tar.gz}{\LaTeX~Template}.
(\href{https://www.asplos2018.org/wp-content/uploads/2017/07/asplos18-template.pdf}{This
  document} was prepared with that template.)  If you use a different
software package to typeset your paper, then please adhere to the
guidelines given in Table~\ref{table:formatting}.\footnote{One
  exception is that authors may use the SIGPLAN style/class file
  \href{http://classic.sigplan.org/sigplanconf.cls}{here}, but {\bf
    only with the 10pt body font option (9pt will be rejected)} and
  modified as needed for the requirements of the references section
  below.  This is marginally different from the specified template,
  but will be accepted due to its widespread use.}

\begin{table}[h!]
  \centering
  \begin{tabular}{|l|l|}
    \hline
    \textbf{Field} & \textbf{Value}\\
    \hline
    \hline
    File format & PDF \\
    \hline
    Page limit & 11 pages, {\bf not including}\\
               & {\bf references}\\
    \hline
    Paper size & US Letter 8.5in $\times$ 11in\\
    \hline
    Top margin & 1in\\
    \hline
    Bottom margin & 1in\\
    \hline
    Left margin & 0.75in\\
    \hline
    Right margin & 0.75in\\
    \hline
    Body & 2-column, single-spaced\\
    \hline
    Separation between columns & 0.25in\\
    \hline
    Body font & 10pt\\
    \hline
    Abstract font & 10pt, italicized\\
    \hline
    Section heading font & 12pt, bold\\
    \hline
    Subsection heading font & 10pt, bold\\
    \hline
    Caption font & 9pt, bold\\
    \hline
    References & 8pt, no page limit, list \\
               & all authors' names\\
    \hline
  \end{tabular}
  \caption{Formatting guidelines for submission. }
  \label{table:formatting}
\end{table}

\textbf{Please ensure that you include page numbers with your
submission}. This makes it easier for the reviewers to refer to different
parts of your paper when they provide comments.

Please ensure that your submission has a banner at the top of the title
page, similar to
\href{https://www.asplos2018.org/wp-content/uploads/2017/07/asplos18-template.pdf}{this
one}, which contains the submission number and the notice of
confidentiality.  If using the template, just replace XXX with your
submission number.

\subsection{Content}

\noindent\textbf{Author List.}  Reviewing will be \textbf{double blind};
therefore, please \textbf{do not include any author names on any submitted
documents except in the space provided on the submission form}.  You must
also ensure that the metadata included in the PDF does not give away the
authors. If you are improving upon your prior work, refer to your prior
work in the third person and include a full citation for the work in the
bibliography.  For example, if you are building on {\em your own} prior
work in the papers \cite{nicepaper1,nicepaper2,nicepaper3}, you would say
something like: "While the authors of
\cite{nicepaper1,nicepaper2,nicepaper3} did X, Y, and Z, this paper
additionally does W, and is therefore much better."  Do NOT omit or
anonymize references for blind review.  There is one exception to this for
your own prior work that appeared in IEEE CAL, workshops without archived
proceedings, etc.\, as discussed later in this document.

\noindent\textbf{Figures and Tables.} Ensure that the figures and tables
are legible.  Please also ensure that you refer to your figures in the main
text.  Many reviewers print the papers in gray-scale. Therefore, if you use
colors for your figures, ensure that the different colors are highly
distinguishable in gray-scale.

\noindent\textbf{References.}  There is no length limit for references.
{\bf Each reference must explicitly list all authors of the paper.  Papers
not meeting this requirement will be rejected.} Authors of NSF proposals
should be familiar with this requirement. Knowing all authors of related
work will help find the best reviewers. Since there is no length limit
for the number of pages used for references, there is no need to save space
here.

\section{Paper Submission Instructions}

\subsection{Declaring Authors}

Declare all the authors of the paper up front. Addition/removal of authors
once the paper is accepted will have to be approved by the program chair,
since it potentially undermines the goal of eliminating conflicts for
reviewer assignment.

\subsection{Areas and Topics}

ASPLOS emphasizes multidisciplinary research. Submissions should ideally
emphasize synergy of two or more ASPLOS areas: architecture, programming
languages, operating systems, and related areas (broadly
interpreted). Authors should indicate these areas on the submission form as
well as specific topics covered by the paper for optimal reviewer match. If
you are unsure whether your paper falls within the scope of ASPLOS, please
check with the program chair -- ASPLOS is a broad, multidisciplinary
conference and encourages new topics.

\subsection{Declaring Conflicts of Interest}

Authors must register all their conflicts on the paper submission site.
Conflicts are needed to ensure appropriate assignment of reviewers.
If a paper is found to have an undeclared conflict that causes
a problem OR if a paper is found to declare false conflicts in order to
abuse or ``game'' the review system, the paper may be rejected.

Please declare a conflict of interest (COI) with the following people
for any author of your paper:

\begin{enumerate}
\item Your Ph.D. advisor(s), post-doctoral advisor(s), Ph.D. students,
      and post-doctoral advisees, forever.
\item Family relations by blood or marriage, or their equivalent,
      forever (if they might be potential reviewers).
\item People with whom you have collaborated in the last five years, including
\begin{itemize}
\item co-authors of accepted/rejected/pending papers.
\item co-PIs on accepted/rejected/pending grant proposals.
\item funders (decision-makers) of your research grants, and researchers
      whom you fund.
\end{itemize}
\item People (including students) who shared your primary institution(s) in the
last five years.
\end{enumerate}

``Service'' collaborations such as co-authoring a report for a professional
organization, serving on a program committee, or co-presenting
tutorials, do not themselves create a conflict of interest.
Co-authoring a paper that is a compendium of various projects with
no true collaboration among the projects does not constitute a
conflict among the authors of the different projects.

On the other hand, there may be others not covered by the above	with whom
you believe a COI exists, for example, close personal friends.
Please report such COIs; however, you may be asked to justify them.
Please be reasonable.	For example, you cannot declare a COI with a
reviewer just because that reviewer works on topics similar to or
related to those in your paper.
The PC Chair may contact co-authors to explain a COI whose origin is unclear.

We hope to draw most reviewers from the PC and the ERC, but others from the
community may also write reviews.  Please declare all your conflicts (not
just restricted to the PC and ERC).  When in doubt, contact the program
chair.


\subsection{Optional Reviewer Suggestions}

Authors may optionally mark (non-conflicted) PC and ERC members that they
believe could provide expert reviews for their submission.  If authors
believe there is insufficient expertise on the PC and ERC for the topic of
their paper, they may suggest alternate reviewers.  The program chair will
use the authors' input at her discretion.  We provide this opportunity
for input mostly for papers on non-traditional and emerging topics.


\subsection{Concurrent Submissions and Workshops}

By submitting a manuscript to ASPLOS'18, the authors guarantee that the
manuscript has not been previously published or accepted for publication in
a substantially similar form in any conference, journal, or workshop. The
only exceptions are (1) workshops without archived proceedings such as in
the ACM digital library (or where the authors chose not to have their paper
appear in the archived proceedings), or (2) venues, such as IEEE CAL, where
there is an explicit policy that such publication does not preclude longer
conference submissions. These are not considered prior publications. 
Technical reports and papers posted on public social media sites, Web pages,
or online repositories, such as arxiv.org, are not considered prior
publications either. In such exceptional cases, the submitted manuscript may
ignore the above work to preserve author anonymity. This information must,
however, be provided on the submission form -- the program chair(s) will
make this information available to reviewers if it becomes necessary to
ensure a fair review. (This policy will be explicitly conveyed to the
reviewers as well.)  The authors also guarantee that no paper that contains
significant overlap with the contributions of the submitted paper will be
under review for any other conference, journal, or workshop during the
ASPLOS'18 review period. Violation of any of these conditions will lead to
rejection.  As always, if you are in doubt, it is best to contact the
program chair(s).  Finally, we also note that the ACM Plagiarism Policy
(http://www.acm.org/publications/policies/plagiarism\_policy) covers a range
of ethical issues concerning the misrepresentation of other works or one's
own work.


%By submitting a manuscript to ASPLOS'18, the authors guarantee that the
%manuscript has not been previously published or accepted for publication in
%a substantially similar form in any conference, journal, or the archived
%proceedings of a workshop (e.g., in the ACM digital library) -- see
%exceptions below. The authors also guarantee that no paper that contains
%significant overlap with the contributions of the submitted paper will be
%under review for any other conference or journal or an archived proceedings
%of a workshop during the ASPLOS'18 review period. Violation of any of these
%conditions will lead to rejection.
%
%The only exceptions to the above rules are for the authors' own papers in
%(1) workshops without archived proceedings such as in the ACM digital
%library (or where the authors chose not to have their paper appear in the
%archived proceedings), or (2) venues such as IEEE CAL where there is an
%explicit policy that such publication does not preclude longer conference
%submissions. These are not considered prior publications.  Technical reports
%and papers posted on public social media sites, Web pages, or online
%repositories, such as arxiv.org, are not considered prior publications
%either. In such exceptional cases, the submitted manuscript may ignore the
%above work to preserve author anonymity. This information must, however, be
%provided on the submission form -- the PC chair will make this information
%available to reviewers if it becomes necessary to ensure a fair review.
%(This policy will be explicitly conveyed to the reviewers.)
%
%As always, if you are in doubt, it is best to contact the program chairs.
%
%Finally, we also note that the ACM Plagiarism Policy ({\em
%http://www.acm.org/publications/policies/plagiarism\_policy}) covers a
%range of ethical issues concerning the misrepresentation of other works or
%one's own work.

\section{Early Access in the Digital Library}

The ASPLOS'18 proceedings will be freely available via the ACM Digital
Library for up to two weeks before and up to a month after the
conference. {\bf Authors must consider any implications of this early
disclosure of their work {\em before} submitting their papers.}


\section{Acknowledgements}

This document is modified from the ASPLOS'17 \href{http://novel.ict.ac.cn/ASPLOS2017/files/asplos17-template.pdf}{submission guide}.

\bibliographystyle{plain}
\bibliography{references}


\end{document}

