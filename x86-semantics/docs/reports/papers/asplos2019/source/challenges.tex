\section{Challenges}

\subsection{Using Strata Results}
 Following are the challenges in using \Strata{}~\cite{Heule2016a} (or \Stoke{}) 
 formula as is.    
 
\begin{itemize}

  \item

    \Stoke uses C+-functions which define the semantics of instructions. For
    example, following is the function to define the semantics of add
    instruction. The functions are generic in the sense that they can be used to
    obtain obtain the concrete semantics of any instruction like \texttt{add\
      \%rax,\ \%rbx}
  
\begin{lstlisting}[language=C]

void add(SymBitVector dest, SymBitVector a, SymBitVector b) {
  set(d, a+b);
}
  \end{lstlisting}
  
  The untested assumption here is the generic formula will behave identically
  for all the variants. We have tested all the formula for each instruction
  variant. 
  
  \item 

  \Strata gives the concrete semantics for a concrete instructions. For other
  variants it generalize from the concrete semantics. Assumption is the
  generalization is correct. Test all the generalization.

  \item 

While porting to \K rule, we generalize the from a concrete semantics that
strata provides. Is this generalization faithful? For instruction like
\instr{xchg, xadd, cmpxchg}, the formula is different for different
operands. So the general \K rule we obtain from \instr{xchgl a, b} may not
represent the semantics for \instr{xchgl a, a}. Fortunately there exists
different instruction variants if the their semantics might be different and
accordingly we might have different \K rules. For example,
            \opcode{xchgl\_r32\_eax} and \opcode{xchgl\_r32\_r32}. But even for
            \opcode{xchgl\_r32\_32} semantics could be different for cases
            \cond{r1\ !=r2} and \cond{r1\ ==\ r2}. Idea: Once lifted as \K
            rule, test the instruction for all variants.  
            
\end{itemize}

Lets consider \instr{xaddb SRC, DEST}, as per manual the semantics is

\begin{eqnarray*}
  \TS{Temp} &=& \TS{Src} + \TS{Dest} \\
  \TS{Src} &=& \TS{Dest} \\
  \TS{Dest} &=& \TS{Temp} \\
\end{eqnarray*}

The point to note here is that the register updates follow an order.  \Strata
uses \instr{xaddb \%rax, \%rbx}, to obtain the semantics and it happened that
the ordering is maintained and hence strata can generalize the semantics of
\instr{xaddb R1, R1}. But even if the ordering is not maintained the
semantics is going to be the same for the case \cond{R1 != R2}, but the generalization
for the \texttt{R1\ ==\ R1} case will mess up.  We cannot trust the above
generalization by strata. We need to test the \K rule for all possible operands.

