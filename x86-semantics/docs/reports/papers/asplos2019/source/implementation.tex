
\section{X86-64 Instruction Semantics in \K} \label{sec:modelI}

\subsection{Modeling Instruction Semantics}
 
\begin{figure}[t]
    \centering
    \begin{tikzpicture}
    [sibling distance=10em,
    every node/.style = {shape=rectangle, rounded corners,
        draw, align=center,
        top color=white\cmt{, bottom color=blue!20}}]]
    
    \node {Register\\ \RegTOTAL{}}
        child { node {Strata-Supported \\ \RegSTRAT{}} }
        child { node {Stoke-Supported \\ \RegSTOK{}} }
        child { node {Manually-Written \\ \RegMAN{}} };
    \end{tikzpicture}
    \caption{Distribution of register instructions}
    \label{fig-reg-dist}
\end{figure}


\tikzset{
    basic/.style  = {draw, text width=2cm, font=\sffamily, rectangle},
    root/.style   = {basic, rounded corners, thin, align=center\cmt{, fill=green!30}},
    level 2/.style = {basic, rounded corners, thin,align=center,  text width=8em},
    level 3/.style = {basic, thin, align=left, text width=6.5em}
}

\cmt{
\begin{figure*}[t]
    \centering
    \begin{tikzpicture}[
    level 1/.style={sibling distance=40mm},
    edge from parent/.style={->,draw},
    >=latex]
    
    % root of the the initial tree, level 1
    \node[root] {Immediates \\ \ImmTotal{}}
    % The first level, as children of the initial tree
    child {node[level 2] (c1) {Generalized}}
    child {node[level 2] (c1) {Generalized}}
    child {node[level 2] (c2) {Non-Generalized}};
    %child {node[level 2] (c3) {Drawing arrows between nodes}};
    
    
    \end{tikzpicture}
    \caption{Distribution of immediate instructions}
    \label{fig-imm-dist}
\end{figure*}
}

\begin{figure}[t]
\centering
\begin{tikzpicture}[
level 1/.style={sibling distance=40mm},
edge from parent/.style={->,draw},
>=latex]

% root of the the initial tree, level 1
\node[root] {Immediates \\ \ImmTotal{}}
% The first level, as children of the initial tree
child {node[level 2] (c1) {Generalized}}
child {node[level 2] (c2) {Non-Generalized}};
%child {node[level 2] (c3) {Drawing arrows between nodes}};

% The second level, relatively positioned nodes
\begin{scope}[every node/.style={level 3}]
\node [below of = c1, xshift=15pt] (c11) {Strata Supported};
\node [below of = c11] (c12) {Stoke Supported};
\node [below of = c12] (c13) {Manually Added};

\node [below of = c2, xshift=15pt] (c21) {Strata Supported};
\node [below of = c21] (c22) {Stoke Supported};
\node [below of = c22] (c23) {Manually Added};
\end{scope}

% lines from each level 1 node to every one of its "children"
\foreach \value in {1,2,3}
\draw[->] (c1.195) |- (c1\value.west);

\foreach \value in {1,2,3}
\draw[->] (c2.195) |- (c2\value.west);

\end{tikzpicture}
\caption{Distribution of immediate instructions}
\label{fig-imm-dist}
\end{figure}

In this work we supported formal semantics of the input/output behavior of
\supp{} out of \total{} x86-64 Haswell ISA instruction variants. Figure \ref{fig:IC} shows the classification of the instructions not supported using dotted ovals. \revisit{Why they are not supported??} 


In order to get semantics of individual instructions, we build on top of project
\Strata~\cite{Heule2016a} which automatically synthesized formal semantics  of
the input/output behavior for $1796$ Haswell ISA X86-64 instructions. The key to
their results is stratified synthesis, where they use a set of instructions
whose semantics are known to synthesize the semantics of additional instructions
whose semantics are unknown. Using a  combination of stochastic search + pruning
using testing (we refer as {\tt initial search}) and subsequent refining of the
search results using equivalence checking ({\tt referred as secondary
    searches}), they first came up with the semantics of $692$ register and
\revisit{$\sim120$} immediate instructions. The rest $\sim984$ are the immediate
and memory variants obtained by generalization of $692$ register instructions.     




Following are some of the immediate challenges that we needed to address.

\begin{enumerate}
    
    \item \textbf{CH.1} Finally how to support the unsupported or
    \emph{un-stratified} ones. The paper ~\cite{Heule2016a} mentions that adding
    some primitive instructions (like saturated add) as the base instruction
    might help stratified more instructions. We would like to explore similar
    directions. Moreover, it would interesting if we can leverage the manually
    written instruction semantics from project \Stoke. \cmt{ and make sure to
        have the same correctness guarantee that \Strata provided in most of its
        cases.} 
           
    \item \textbf{CH.2} The $\sim120$ immediate instructions, mentioned above,
    do not have a corresponding register-only instruction to generalized from.
    Therefore \Strata tries to learn a separate formula for every possible value
    of the 8 bit immediate operand.  We intend to have a more intuitive generic
    semantics (that works for all values of the immediate operand) for those
    instructions. 

 \item \textbf{CH.3} There are instructions which conditionally sets some cpu
 flags to \emph{undef}. For example, the shift left instruction \instr{salq
   \%cl, \%rbx} sets flag \reg{of} to \emph{undef} state if the count mask $>1$.
   Also there are instructions like \instr{blsr \%eax, \%ebx} which
   un-conditionally puts flags like \reg{pf} \& \reg{af} into \emph{undef}
   state.
    
    \Strata while doing the {\tt initial search} does not test the flags which
    \emph{may} (for conditional \emph{undef}s)  or \emph{must} (for
        un-conditional \emph{undef}s) be taking undefined values. We intend to
    model the semantics of these flags with the same correctness guarantee as
    the other registers which do not result in \emph{undef} and hence modeled by
    \Strata.

    \item \textbf{CH.4} \Strata chose not to model the \reg{af} flag as this is
    not commonly used. Supporting this flag fall within the scope of our work.
      
    \item \textbf{CH.5} How reliable is the generalization of register
    instructions to memory or immediate variants? \Strata states that the claim
    for the generalization is based on random testing.
    
        

\end{enumerate} 

Following is a key observation concerning stratification which help us handle
the most of the above mentioned challenges.

\paragraph{Observation} In order to get the semantics of a target instruction
{\tt I}, \Strata uses \Stoke along with a set {\tt TS} of $6580$ test cases to
synthesize an instruction sequence which agrees with {\tt I} on {\tt TS} (which
    means the output behavior of the instruction sequence matches with that on real
    hardware for input {\tt TS}). After having that {\tt initial
  search}, they keep on searching  additional sequences, called {\tt
      secondary searches}, each agreeing with {\tt I} on {\tt TS}, in a hope of
  getting  one which would prove non-equivalent to existing ones and thereby
  gaining more confidence on the search and probably an augmented test-suite (as
      {\tt TS} might get augmented with a counter example from equivalence
      checker in the event of non-equivalence). 
      
      One unfavorable possibility for
  \Strata is when all subsequent secondary search results proves  equivalent to
  the one obtained from initial search and hence there are no conflicts among searches, in which case it  means that  secondary
  searches fail to add any ``confidence'' to the initial search result and end up giving the same correctness guarantee as provided by the initial search result. Even though in such unfavorable case, the secondary searches might have
    provided ``better'' choices to pick the final formula from. A better choice
    of formula do not contain uninterpreted functions or  non-linear arithmetics
    and are simple.  
    
   In the paper\cite{Heule2016a}, it is mentioned that there are only $50$
   cases, where they found a (valid) counterexample. That means, there are $762
   = (692 + 120 - 50)$ instructions, whose the initial stoke search using augmented test-suite, containing $6630\ (= 6580 + 50$) tests,  is sufficient enough to provide a semantics with the same correctness guarantee which \Strata provides.   In other words, in  most of
   the cases, the correctness guarantee of secondary searches is same as that of the initial stoke search using the augmented test-suite (henceforth referred as {\tt ATS})  which \Strata ends up with.
      
      
    \paragraph{Handling CH.1}
   For an unsupported instruction {\tt I}, we either model its semantics  manually or borrowed it from project \Stoke.   
   Once we have this candidate, we test it against hardware using {\tt ATS}.
   Once the test passes we claim (from the above observation) the
   semantics to have the same correctness guarantee which \Strata provides for 
   most of its cases.    
    
    We understand that this
   is not as efficient as \Stoke, which is fully automatic in getting these
   formulas, and we do not intend to make any contribution towards efficient 
   generation of instruction semantics. The purpose of above mentioned effort is to deliver in cases where \Stoke cannot without loosing 
   much on the correctness guarantee. 
   
   Moreover, writing the
   semantics manually might alleviate the need of secondary search as a means to
   provide ``better'' formula as we can control the complexity and choice of
   operations to include in the formula. Also carefully written manual formula
   tend to need less number of conflicting searches than the onces generated by
   random search engines like Stoke.
   
   \revisit{add base + reduce search space}
   
   
   \paragraph{Handling CH.2} The instructions in this category either  have a
   separate formula for all or some of 256 possible values. We refer each of the
   separate formulas for instruction {\tt I} as a concrete formula \CF{I}{c} for
   a particular constant value {\tt c} of immediate operand.  In either case, we
   get a generic formula, $G^I$ either by writing it manually or borrowing it
   from  \Stoke project.
   
   In the case where we have a separate \CF{I}{c} $\forall c \in \{0...255\}$,
   we do a Z3 equivalence check as follows: $\forall c \in \{0..255\}:$
   \CF{I}{c} $\equiv_\text{Z3}$ \GN{I}{c}, where \GN{I}{c} is obtained by
   replacing the symbolic inputs of $G^I$ with constant value {\tt c}. A
   successful equivalence check suggest $G^I$ to be a generic formula with the
   same correctness guarantee that \Strata has for any of the individual
   concrete formulas. For the case where we have a separate formula for a subset
   of constant values, we do the same equivalence check as before for that
   subset. The constants for which we do not have a separate formula we test
   \GN{I}{c} using {\tt TS}, the final test-suite of \Strata. 
    
    \paragraph{Handling CH.3} There are $474\ (= 141(\text{Reg}) +
        109(\text{Imm}) + 224(\text{Mem}))$ instructions that results in
    conditional  (or \emph{may}) \emph{undef} ($162\ (= 40 + 46 + 76)$) or
    un-conditional (or \emph{must}) \emph{undef} ($312\ (= 101 + 63 + 148)$)  in
    one or more cpu flag.  The semantics of most of the cpu flags (which
        \emph{may} or \emph{must} take \emph{undef} values) are already modeled
    in \Stoke. We needed to model the semantics of flag registers for $40$
    instructions involving {\tt shifts, rotates}~\cite{BugStoke986}. 
    
    For \emph{may undef} cases, we tested against hardware, using {\tt TS}, for
    the scenarios when the condition for undefinedness is not triggered.  For
    the remaining cases, (1) \emph{may undef}s where the condition for
    triggering \emph{undef} is true and (2) \emph{must undef}, we make sure that
    \K execution halts when the undefinedness condition is triggered. This help
    is find bugs in the \Stoke implementation of $8$
    instructions~\cite{BugStoke986} ( Note that these $8$ instructions are not
        stratified and hence we borrowed it from \Stoke).   
    
    
   \paragraph{Handling CH.4}
   \begin{figure}[t]
       \centering
       \fig{figures/af_distribution.eps}
       \caption{Instructions affecting \reg{af} flag.} The numbers represent count of (Register/Immediate/Memory) Instructions. 
       \label{fig:AD}
   \end{figure}

   Figure \ref{fig:AD} represents the distribution of instructions affecting the
   \reg{af} flag in a defined or un-defined way (which could be conditional or
       un-conditional).  We tested all the instructions for the defined cases
   using {\tt TS}. For conditionally undefined cases, we tested for the
   scenarios when the condition for undefinedness is not triggered.  For all
   remaining cases,  we make sure that the \K execution halts when the
   undefinedness condition is triggered.        
   
   
   
\subsection{Porting to \K Rules}

For the purpose of getting  \K rules, we could have directly converted the
\Strata formulas for an instruction to \K rule assuming that the \Strata's
symbolic execution over the stratified instruction sequence is correct.

Given that fact the \K's symbolic execution engine is more trusted as that has
been used extensively in language-agnostic manner to perform symbolic execution,
     we decided to use \K's symbolic executor. Also in order to check if
     \Strata's symbolic execution engine is correct, we did an equivalence check
     on the outputs of both the symbolic executions.   
 

\begin{enumerate}
\item Implementing the base instructions semantics in \K and testing them.
\item Symbolic execution of the stratified instruction sequences.
\item Dealing with scratch pad registers.
\item Equivalence check between \Strata formula ad the output of 2.
   
   All the checks are \emph{unsat}, expect one where the check fail to due a bug
   in the simplification rules in \Strata, which states the following lemma
   related to two single precision floating point numbers  {\tt A}  and {\tt B},
   which is not correct for {\tt NaNs}. However this bug is fixed in the latest
   version of \Stoke. 
   
   
   { \tt  
        \begin{tabular}[b]{l}
   \qquad sub\_single(A, B) $\equiv$ 0 if A == B     
      \end{tabular}
  }
   
\item {Simplification of formulas:} Simplification generates simple \K rule
(sometimes simpler than the corresponding \Strata formula).  Also it is much
easier to write the simplification rules in \K.\revisit{show the example for
  concat(A[1:2], concate(B[2:3], X)) $\equiv$ concate(A[1:3], X)}


\item One drawback of the \Strata formulas is they could be non-intuitive and
complex at times when the simplification rules are not adequate enough to
simplify their complexity to more intuitive formulas. Appendix \ref{sec:AP:A}
provides such an example.  Towards the goal of having intuitive formulas, we
borrowed the hand written formula (provided they are simpler) from \Stoke or
manually write those  and check equivalence with the stratified formula. If they
match on all register state and/or memory, we employ that in our \K semantics.

\cmt{
An example of one such simplification opportunity is: 
     { \tt  
      \begin{tabular}[b]{l}
          ($0_{32}$ $\cdot$ \%rax[32:0]) $\oplus$ \%rax $\equiv$ \%rax[63:32] $\cdot$  $0_{32}$ 
      \end{tabular}
    }
}

       


\end{enumerate}

\subsection{Supporting un-stratified instructions \& Porting their formulas to \K Rules}

\subsubsection{Supporting un-stratified instructions}
\paragraph{Instruction support status}



\subsection{Porting to \K Rules}

\Strata could output the internal AST, used to model a register state formula, in different
formats. Supported backend are SmtLib and Prefix notation. We have added another backend 
to generate \K rule. We need some way to validate the backend. 

\paragraph{Validate the Backend}

The \K rules generated using the backend are matched (syntactically)  against
the ones we already obtained via symbolic execution on stratified instructions.
Other than validaing the backend, this has an added benefit that in order to get
the exact match, we need to port all the simplification rules from \K to strata
code, which in turn will later help in generating simplified \K rules for
non-stratified instructions. 

Main challenges in getting an exact match are:
\begin{itemize}

\item  \Strata rules uses \extract to extract portion of a bit-vector. The high
and low indices of \extract are obtained considering LSB at index 0, whereas \K
uses \extractMInt for the same purpose, but uses MSB at index zero.

\item  \Strata uses flags as \bool, whereas they are treated as \bv in our
semantics. We modifed strata so as to treat flag registers as 1 bit bitvectors.

\end{itemize}


